%TEX root = ./overwiew.tex

\documentclass{standalone}
% http://texample.net/tikz/examples/inertial-navigation-system/

\usepackage{tikz}
\usetikzlibrary{positioning}
\usetikzlibrary{shapes,arrows}
\usetikzlibrary{arrows.meta, positioning, quotes}

\begin{document}

\tikzstyle{module}=[draw, fill=blue!20, text width=5em, text centered, minimum height=2cm]
\tikzstyle{module_def}=[fill=yellow!20,rounded corners, draw=black!50, dashed]
\tikzstyle{signal_name} = [above, text width=10em]
\tikzstyle{module_name} = [above right, text width=10em]
\tikzstyle{spn_node} = [draw, circle, fill=green!20, minimum width=0.5cm]
\tikzstyle{sy} = [yshift=#1mm]

% \def\blockdist{3.5cm}
% \def\edgedist{1.5cm}

\pgfdeclarelayer{background}
\pgfdeclarelayer{foreground}
\pgfsetlayers{background,main,foreground}

\begin{tikzpicture}

	\node (input) [module, minimum width=2cm, minimum height=8cm] {};
	\node () [rotate=90] at (input) {Input module};

	\node (spn)[module, minimum width=6cm, minimum height=6cm, above right=-6cm and 1cm of input] {};
	\node [below right=-0.7cm and -1cm of spn] {SPN};


	\node (ctrl)[module,minimum width=2cm,minimum height=2.2cm,above right=-2.2cm and 4cm of spn] {Control};
	\node (mem)[module, minimum width=2cm,minimum height=4cm,below=1.5cm of ctrl] {RAM};

	\node (fifo)[module, minimum width=1cm, minimum height=1.5cm, above left=-1.8cm and 1.5cm of mem] {FIFO};


	\begin{pgfonlayer}{foreground}
		\node (A) [spn_node, above left=-1cm and -1cm of spn] {$A$};
		\node (notA) [spn_node, below =0.7cm of A] {$\bar{A}$};
		\node (B) [spn_node, below =0.7cm of notA] {$B$};
		\node (notB) [spn_node, below =0.7cm of B] {$\bar{B}$};

		\node (plus1) [spn_node, right=1cm of A] {$+$};
		\node (plus2) [spn_node, right=1cm of notA] {$+$};
		\node (plus3) [spn_node, right=1cm of B] {$+$};
		\node (plus4) [spn_node, right=1cm of notB] {$+$};

		\node (x2) [spn_node, right=-2.4cm of spn] {$\times$};
		\node (x1) [spn_node, above=0.8cm of x2] {$\times$};
		\node (x3) [spn_node, below=0.8cm of x2] {$\times$};

		\node (plus) [spn_node, right=-1cm of spn] {$+$};

		\path [draw, ->] (A) -- node[signal_name, align=center] {$w_i$} (plus1);
		\path [draw, ->] (notA) -- node[signal_name, align=center] {} (plus1);
		\path [draw, ->] (A) -- node[signal_name, align=center] {} (plus2);
		\path [draw, ->] (notA) -- node[signal_name, align=center] {} (plus2);

		\path [draw, ->] (B) -- node[signal_name, align=center] {} (plus3);
		\path [draw, ->] (notB) -- node[signal_name, align=center] {} (plus3);
		\path [draw, ->] (B) -- node[signal_name, align=center] {} (plus4);
		\path [draw, ->] (notB) -- node[signal_name, align=center] {} (plus4);

		\path [draw, ->] (plus1) -- node[signal_name, align=center] {} (x1);
		\path [draw, ->] (plus1) --  (x2);
		\path [draw, ->] (plus4) --  (x3);
		\path [draw, ->] (plus4) --  (x2);
		\path [draw, ->] (plus2) --  (x3);
		\path [draw, ->] (plus3) --  (x1);

		\path [draw, ->] (x1) -- node[signal_name, align=center] {$w_i$} (plus);
		\path [draw, ->] (x2) --  (plus);
		\path [draw, ->] (x3) --  (plus);

	\end{pgfonlayer}

    \begin{pgfonlayer}{background}
    	\path (input.west |- input.north)+(-0.2,0.2) node (a1) {};
    	\path (fifo.east |- input.south)+(0.5,-0.2) node (b1) {};
        \path [module_def] (a1) rectangle (b1);
        \path (b1) node[module_name, above] (name) {FPGA};

        \path (ctrl.west |- ctrl.north)+(-0.2,0.2) node (a) {};
    	\path (ctrl.east |- ctrl.south)+(0.2,-0.5) node (b) {};
        \path [module_def] (a) rectangle (b);
        \path (b) node[module_name, above right=0cm and -1cm] (name) {OS};

        \path (mem.west |- mem.north)+(-0.2,0.2) node (a) {};
    	\path (mem.east |- mem.south)+(0.2,-0.5) node (b) {};
        \path [module_def] (a) rectangle (b);
        \path (b) node[module_name, above right=0cm and -1.7cm] (name) {Memory};
    \end{pgfonlayer}


	% From memory to input
	\path [draw, ->] (mem.235) -- (input.east |- mem.235) node[signal_name, above=0cm, midway] {Fixed address read};

	% From input to SPN
	\coordinate[below=of input.north east] (sig1);
	\foreach \i [count=\j] in {-1,0,1,2,3,4,5,6,7,8,9}
	{
	    \draw[->]   ([sy=-\i*5] sig1) to ([sy=-\i*5] sig1 -| spn.west);
	}

	% From SPN to output
	\path [draw, ->] (spn.-30) -- (fifo.west);

	\path [draw, ->] (fifo.east) -- (mem.west |- fifo.east);

	% Control to RAM and SPN
	\path [draw, ->] (ctrl.-120) -- (ctrl.-120 |- mem.north);
	\path [draw, <-] (ctrl.-90)  -- (ctrl.-90 |- mem.north) node[signal_name, align=center, right=0.4cm, midway, below] {AXI};
	% \path [draw, ->] (ctrl.160) -- (b1 |- ctrl.160);
	% \path [draw, <-] (ctrl.-160) -- (b1 |- ctrl.-160);

	\path [draw, ->] (ctrl.40) --  ++(1,0) node[signal_name, right=0cm] {clk\_spn};
	\path [draw, ->] (ctrl.15) --  ++(1,0) node[signal_name, right=0cm] {clk\_input};
	\path [draw, ->] (ctrl.-15) -- ++(1,0) node[signal_name, right=0cm] {stdout};

	\path [draw, <-] (input.107) -- ++(-0.8,0) node[signal_name, left=-2cm] {clk\_input};
	\path [draw, <-] (spn.125) -- ++(0,0.5) -- ++(-4.9, 0) node[signal_name, left=-2cm]{clk\_spn};


\end{tikzpicture}

\end{document}

