%TEX root = ./overwiew.tex

\documentclass{standalone}
% http://texample.net/tikz/examples/inertial-navigation-system/

\usepackage{tikz}
\usetikzlibrary{positioning}
\usetikzlibrary{shapes,arrows}

\begin{document}

% \tikzstyle{module}=[draw, fill=blue!20, text width=5em, text centered, minimum height=2cm]
% \tikzstyle{module_def}=[fill=yellow!20,rounded corners, draw=black!50, dashed]
% \tikzstyle{signal_name} = [above, text width=10em]
% \tikzstyle{module_name} = [above right, text width=10em]

\tikzstyle{base}=[draw, circle, minimum height=1cm]

\def\blockdist{3.5cm}
\def\edgedist{2cm}

\begin{tikzpicture}

    \node (literal) [base] {L};
    \node (val) [above right=0cm and 0cm of literal] {\color{gray} Id};
    \node (X) [below =of literal] {$X_i$};
    \node () [above right=0cm and 0cm of X] {\color{gray} Lit};

    \path [draw, ->] (literal) -- (X);
    % \node (encoder) [module,minimum height=4cm] {Posit\\Encoder};
    % \path (encoder.150)+(-\blockdist,0) node (exponent) [module] {Exponent\\encoder};
    % \path (encoder.-130)+(-2*\blockdist,0) node (input) [] {};
    % \path (encoder.0)+(0.5*\blockdist,0) node (output) [] {};

\end{tikzpicture}

\end{document}

